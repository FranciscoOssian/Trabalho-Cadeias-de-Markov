\documentclass[a4paper,12pt]{article}
\usepackage[utf8]{inputenc}

% para separar silabas em portugues use
\usepackage[brazil]{babel}

% para indentar o primeiro parágrafo
\usepackage{indentfirst}

% para usar fontes e símbolos da AMS
\usepackage{amsfonts}
\usepackage{amssymb}
\usepackage{amsmath}

% para ambientes do tipo theorem
\newtheorem{proposition}{\scshape Proposição}[section]
\newtheorem{corollary}{\scshape Corolário}[section]
\newtheorem{lemma}{\scshape Lema}[section]
\newtheorem{definition}{\scshape Definição}[section]
\newtheorem{conjecture}{\scshape Conjectura}[section]
\newtheorem{example}{\scshape Exemplo}

% alguns atalhos para escrever vetores em negritoyhyhyhyhyhhyyhyhyhhy
\renewcommand{\u}{{\bf u}}
\renewcommand{\v}{{\bf v}}
\renewcommand{\sin}{\operatorname{sen}}
\providecommand{\w}{{\bf w}}
\providecommand{\x}{{\bf x}}
\providecommand{\y}{{\bf y}}
\providecommand{\q}{{\bf q}}
\providecommand{\bfa}{{\bf a}}
\providecommand{\bfb}{{\bf b}}
\providecommand{\bfc}{{\bf c}}
\providecommand{\zero}{{\bf 0}}
\providecommand{\spn}{\mathrm{span}}
\providecommand{\posto}{\mathrm{posto}}
\providecommand{\nul}{\mathrm{nul}}
\providecommand{\proj}{\mathrm{proj}}
\providecommand{\tr}{\mathrm{tr}}







%opening




\title{$2^o$ Trabalho de Álgebra Linear para Computação}
\author{
  Francisco Ossian Lopes Neto\\
  \and
  xxxxxxxxxxxxxxxxxxxxxxx\\
  \and
  yyyyyyyyyyyyyyyyyyyyyyyy\\
  \and
  zzzzzzzzzzzzzzzzzzzzzzzz\\
  \and
}
\date{abril 2020}

\begin{document}

\maketitle

\begin{abstract}
O resumo deve conter a descrição do problema, a técnica, algoritmo ou método utilizado, a metodologia e uma indicação dos resultados obtidos. Só pode haver um parágrafo no resumo e não é permitido fazer referências bibliográficasxxxxxxxxxxxxxxxxxxxxxxxxxxxxxxxxxxxx.
\end{abstract}


%%%%%%%%%%%%%%%%%%%%
\section{Introdução}
\label{sec:introducao}

Podemos abordar na probabilidade da genética, inumeros casos peculiares que podemos resolve-los atravez da Álgebra Linear. Como quais as proporções em que populações tendem depois de vários períodos, se conparadas pelos seus genes. Sabemos que existem genes que são mais dominates, e por isso exercem uma força maior na velocidade de reprodução. Tendo isto em questão, podemos usar o sistema das Cadeias de Markov para calcular a probabilidade de proporção de determinados grupos em tempos discretos. 


%%%%%%%%%%%%%%%%%%%%%%%%%%%%%%%
\section{Fundamentação Teórica (substitua por um título apropriado relacionado com seu trabalho)xxxxx}
\label{sec:fundamentacao}

Um processo de Markov é um processo estocástico em que a probabilidade de o sistema estar no estado $i$ no período $(n+1)$ depende somente do estado em que o sistema está no período $n$. Ou seja, para os processos de Markov, só interessa o estado imediato. Os principais elementos de um processo de Markov são dois:

\begin{itemize}

\item as probabilidades de transição $Mij$. Estas probabilidades de transição são normalmente agrupadas numa matriz, que denominamos matriz de transição, matriz estocástica ou ainda matriz de Markov. Onde $i$ representa o período $n$ e $j$ o período $n+1$. Ou $j$ o período $n$ e $i$ o período $n+1$

\item a probabilidade $x^i(n)$ de ocorrer o estado $i$ no $n$-ésimo período de tempo, ou, alternativamente, a fração da população em questão que está no estado $i$ no $n$-ésimo período de tempo.

\end{itemize}


%%%%%%%%%%%%%%%%%%%%%
\section{Metodologia e Experimentos}
\label{sec:metodologia}


Consideremos uma matriz $M$, tal que
\begin{displaymath}
M = \begin{bmatrix}
0.5&0.25&0\\
0.5&0.5&0.5\\
0&0.25&0.5
\end{bmatrix}
\end{displaymath}

xxxxxxNesta seção, todo a metodologia usada no trabalho deve ser explicitada. Ou seja, a sequência do que foi feito e os parâmetros usados.

Por exemplo, a matriz de Leslie relativa à população feminina americana em 1967.

Os experimentos consistem, por exemplo, da centralização dos dados na matriz $X$, as matrizes formadas para compor as imagens (equipe C).xxxxxx



%%%%%%%%%%%%%%%%%%%%
\section{Resultados e Conclusão}
\label{sec:resultados}

Nesta seção, aparecem os gráficos, as tabelas etc. Lembre-se que todo gráfico (que estão num ambiente de figura) ou tabela deve ser referenciado e explicado no \textbf{texto}.

A conclusão é um pequeno resumo dos resultados obtidos, por exemplo, como a população de genes dominantes, híbridos e recessivos se estabiliza.

%%%%%%%%%%%%%%%%%%%%%%%%%%%%%%
%%% referencias bibliograficas
\bibliographystyle{apalike}
% \bibliographystyle{abnt-alf}
\bibliography{biblio}

\end{document}

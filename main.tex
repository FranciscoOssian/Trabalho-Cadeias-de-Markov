\documentclass[14pt]{article}
\usepackage[utf8]{inputenc}
\usepackage{tikz}
\usepackage{pgfplots}

\documentclass[a4paper,12pt]{article}
\usepackage[utf8]{inputenc}

% para separar silabas em portugues use
\usepackage[brazil]{babel}

% para indentar o primeiro parágrafo
\usepackage{indentfirst}

% para usar fontes e símbolos da AMS
\usepackage{amsfonts}
\usepackage{amssymb}
\usepackage{amsmath}

% para ambientes do tipo theorem
\newtheorem{proposition}{\scshape Proposição}[section]
\newtheorem{corollary}{\scshape Corolário}[section]
\newtheorem{lemma}{\scshape Lema}[section]
\newtheorem{definition}{\scshape Definição}[section]
\newtheorem{conjecture}{\scshape Conjectura}[section]
\newtheorem{example}{\scshape Exemplo}

% alguns atalhos para escrever vetores em negritoyhyhyhyhyhhyyhyhyhhy
\renewcommand{\u}{{\bf u}}
\renewcommand{\v}{{\bf v}}
\renewcommand{\sin}{\operatorname{sen}}
\providecommand{\w}{{\bf w}}
\providecommand{\x}{{\bf x}}
\providecommand{\y}{{\bf y}}
\providecommand{\q}{{\bf q}}
\providecommand{\bfa}{{\bf a}}
\providecommand{\bfb}{{\bf b}}
\providecommand{\bfc}{{\bf c}}
\providecommand{\zero}{{\bf 0}}
\providecommand{\spn}{\mathrm{span}}
\providecommand{\posto}{\mathrm{posto}}
\providecommand{\nul}{\mathrm{nul}}
\providecommand{\proj}{\mathrm{proj}}
\providecommand{\tr}{\mathrm{tr}}







%opening




\title{$2^o$ Trabalho de Álgebra Linear para Computação}
\author{
  Francisco Ossian Lopes Neto\\
  \and
  Paulo Henrique Souza Filho\\
  \and
  Alexandre Cavalcante de Azevedo\\
  \and
}
\date{abril 2020}

\begin{document}

\maketitle

\begin{abstract}
Probabilidade na genética, envolve diversas vezes, soluções que dependem da álgebra linear para transformar seu problmeas em triviais. Um deles é a taxa de probabilidade genética ao passar das gerações. Sendo este problema resolvido com as Cadeias de Markov que juntam diversas técnicas e métodos matriciais, para representar justamente a taxa de variação ao decorrer das gerações. Utilizando autovalores, autovetores, decomposição espectral, entre outros.
\end{abstract}


%%%%%%%%%%%%%%%%%%%%
\section{Introdução}
\label{sec:introducao}

Podemos abordar na probabilidade da genética, inúmeros casos peculiares que podemos resolver atravéz da Álgebra Linear. Como, quais as proporções em que populações tendem depois de vários períodos, se conparadas pelos seus genótipos. Sabemos que existem genes que são mais dominates, e por isso exercem uma força maior na probabilidade genótipa. Tendo isto em questão, podemos usar o sistema das Cadeias de Markov junto a operações matriciais da Álgebra Linear para calcular a probabilidade da proporção de determinados grupos em tempos discretos. 


%%%%%%%%%%%%%%%%%%%%%%%%%%%%%%%
\section{Fundamentação Teórica}
\label{sec:fundamentacao}

Um processo de Markov é um processo estocástico em que a probabilidade de o sistema estar no estado $i$ no período $(n+1)$ depende somente do estado em que o sistema está no período $n$. Ou seja, para os processos de Markov, só interessa o estado imediato. Os principais elementos de um processo de Markov são dois:

\begin{itemize}

\item As probabilidades de transição $Mij$. Estas probabilidades de transição são normalmente agrupadas numa matriz, que denominamos matriz de transição, matriz estocástica ou ainda matriz de Markov. Onde $i$ representa o período $n$ e $j$ o período $n+1$. Ou $j$ o período $n$ e $i$ o período $n+1$

\item A probabilidade $x^i(n)$ de ocorrer o estado $i$ no $n$-ésimo período de tempo, ou, alternativamente, a fração da população em questão que está no estado $i$ no $n$-ésimo período de tempo.

\end{itemize}

    Para nós encontrarmos autovalores e autovetores de uma matriz $M$, precisamos seguir uma lógica de uma equação:
$M * \vec{v} = \lambda * \vec{v}$,
onde os simbolos representam respectivamente, a matriz de estados de Markov, um autovetor da matriz, um autovalor da matriz, e o autovetor da matriz. \\

    E, sendo um autorvetor $\vec{v}_2$ e um autovalor $\lambda_2$,
\begin{center}
$\lambda_2 I \vec{v}_2 = \lambda_2 \vec{v}_2$
\end{center}

    onde $I$ é a matriz identidade.
Então podemos isolar a matriz de Markov e $\lambda I$, tal que.

\begin{center}
$(M-\lambda I)\vec{v} = \vec{0}$.\end{center}

Fazendo agora o determinate temos, 
$det((M-\lambda I)\vec{v}) = det(\vec{0})$, logo:

\begin{center}
    $det(M - \lambda I) = 0$
\end{center}.

Pois por definição $\vec{v}$ é não nulo, e $det(0) = 0$\\\\
e para conseguir os autovalores, fazemos o determinate, tendo a matriz $M$. No final encontramos um polinômio de grau 3,  $ax^3 + bx^2 + cx^1 + dx^0$. Onde cada raíz deste polinômio é um $\lambda$ (autovalor) da matriz M.\\

E em casos que $\lambda$ é um valor entre $0$ e $1$, com o passar de $n$ períodos, onde $n$ tende a infinito, o resesultado irá tender a $0$, pois\\
\begin{center}
$\lim_{n\to\infty} x^n = 0$\\ tal que,  $0 < n < 1$\end{center}

E calculando o autovalor, utilizamos mais uma vez\\
\begin{center}
    $M \vec{v} = \lambda \vec{v}$,
\end{center}
e substituimos $\lambda$ pelo valor que encontramos, formando um sistema e encontrando o autovetor $\vec{v}$.\\

Uma matriz $M$, pode ser escrita como, $M = S\Lambda S^{-1}$,
onde $\Lambda$ é a matriz diagonal dos autovalores de $M$ e $S$ é uma matriz cujas colunas são os autovetores associados dos autovalores de $M$\\

Sabendo que uma matriz pode ser decompostar em $S\Lambda S^{-1}$, fica trivial o cálculo de sua potência. Sendo:
\begin{center}
    $M^k = (S\Lambda S^{-1})^k = S\Lambda S^{-1} S\Lambda S^{-1} ... S\Lambda S^{-1} = S\Lambda^k S^{-1} $
\end{center}

$ $\\

Seja uma matriz $A_{nxn}$ cuja soma dos elementos de cada coluna é zero, logo
$det(A) = 0$, pois:\\


Sendo $B$ um vetor de dimenção $1 x n$ contendo o somatória dos elementos de cada coluna de A\\

\begin{center}
    $B_{ij} = \sum_{j=0}^{n} Aij$\\
    $det(A) = det(B)$\\
\end{center}

Por definição $B$ é um vetor com uma linha nula, logo $det(B) = 0$, pois em uma matriz ou vetor que contenha um vetor nulo, seu determinante será $0$\\

E seja $K_{nxn}$ uma matriz cuja soma dos elementos de cada coluna é 1, logo existirá um autovalor $\lambda = 1$, pois:\\

\begin{center}
    $det(K - \lambda I) = 0$\\
    $det(K - I) = 0$
\end{center}.

Sendo que a matriz $K-I$ tem o somatório de elementos de cada coluna igual a 0, pois de cada coluna de $K$ é retirado 1. E tendo somatório de cada coluna igual a zero, seu determinate será zero. Satisfazendo a condição para $\lambda = 1$ ser um autovalor.



%%%%%%%%%%%%%%%%%%%%%
\section{Metodologia e Experimentos}
\label{sec:metodologia}


Consideremos uma matriz $M$, tal que $Mij$ representa a transição do estado $i$ para o estado $j$. Fechando exclusivamente a uma transição em que temos no promeiro período um genótipo qualquer e nos $N$ proxímos períodos esse genótipo qualquer seja combinado com um genótipo heterogêneo. Tendo assim apenas 1 probabilidade para cada transição. 

\begin{displaymath}
M = \begin{bmatrix}
0.5&0.25&0\\
0.5&0.5&0.5\\
0&0.25&0.5
\end{bmatrix}
\end{displaymath}

Os experimentos consistem em Encontrar respostas de $n$ períodos das transições de estados, por meio de técnicas matriciais, como autovalores, autovetores e decomposição espectral.\\

A decomposição espectral junto aos autovalores iram simplificar os cálculos da potência da matriz $M$, por meio da diagonalização da mesma. \\

O autovalor predominante da matriz $M$ vai nos dar por conseguinte seu autovetor, sendo este autovetor a proporção de cada genótipos no período $k$ representado pelo $\lim_{k\to\infty}$.\\

Calculando os autovalores, temos:
\begin{center}
    $\lambda_1 = 1$,\\ $\lambda_2$ = 0.5,\\ $\lambda_3 = 0$.
\end{center}.

Pois,
\begin{center}
    $det(M-\lambda I) = 0$\\
    $= (0.5 - \lambda)^3 - 1/16 - \lambda/8 = 0$
\end{center}
e $\lambda_1$, $\lambda_2$, $\Lambda_3$ são suas respectivas raízes.

Como já definido, se um autovalor for entre $0$ e $1$, ele não será considerado um autovalor suficiente para $n$ tendendo ao infinito.

Que no caso da matriz $M$, os limites dos autovalores são :
\begin{center}
    $\lim_{k\to\infty} (\lambda_1)^k = 1$,\\
    $\lim_{k\to\infty} (\lambda_2)^k = 0$,\\
    $\lim_{k\to\infty} (\lambda_3)^k = 0$
\end{center}

Então o autovalor dominante é o $\lambda_1$. E seu autovetor associado é, no caso ao subistituirmos $\lambda_1$ em $M\vec{v} = \lambda \vec{v}$ e calcularmos o sistema objtido, obtemos que o autovetor $\vec{v}$ associado a $\lambda_1$ é igual a:

\begin{displaymath}
\vec{v_1} = \begin{bmatrix}
1\\
2\\
1
\end{bmatrix}
\end{displaymath}

Sendo está a proporção de genes quando o número de períodos tende a infinito.
Mas esta proporção não está representada como probabilidade. Para nós convertermos a proporção de genes para probabilidade, precisamos normalizar o vetor $\vec{v}_1$ de acordo com a razão:\\

\begin{center}
    $\left( 1 / \sum_{i=j}^{3} \vec{v}_{j1} \right)\vec{v}$    
\end{center}

$ $\\

logo:
\begin{center}
    $\left( 1 / \sum_{i=j}^{3} \vec{v}_{j1} \right) = \frac{1}{4}$\\
    $ $\\
    $ $\\
    $\frac{1}{4} \vec{v} =  \begin{bmatrix} 1/4\\ 1/2\\ 1/4 \end{bmatrix} = $
    $\begin{bmatrix} 0.25\\ 0.5\\ 0.25 \end{bmatrix}$
\end{center}

Sendo está a probabilidade sobre cada genótipo, no período $k$ tal que,  $k = \lim_{k\to\infty}$\\

E podemos então calcular a decomposição espetral de $M$, para quando quisermos calcular a proporção de genes em qualquer período de formar viável.

primeiro precisamos calcular todos os autovalores e vetores de $M$, os autovalores já foram calculadores, sendo $\lambda_1 = 1; \lambda_2 = 0.5; \lambda_3 = 0$\\

Então precisamos calcular os autovetores relacionados a cada autovetor, resolvendo 2 sistemas linerares (pois $\vec{v}_1$ já foi calculado):\\

\begin{center}
    $ M\lambda_2 = \lambda_2 \vec{v}_2  $\\
    $ $\\
    $\begin{bmatrix}
    0.5&0.25&0\\
    0.5&0.5&0.5\\
    0&0.25&0.5
    \end{bmatrix} 0.5 = 0.5 \begin{bmatrix} \vec{v}_{21}\\ \vec{v}_{22}\\ \vec{v}_{23} \end{bmatrix}$\\
    $ $\\
    $ $\\
    $\begin{cases}
        \frac{2\vec{v}_{21} \vec{v}_{22}}{4} = 0\\
        \frac{\vec{v}_{21} \vec{v}_{22} \vec{v}_{23}}{2} = 0\\
        \frac{\vec{v}_{22}+2 \vec{v}_{23}}{4} = 0 \end{cases}
    $
    $ $\\
    $ $\\
    logo,\\
    $ $\\
    $\vec{v}_2 = \begin{bmatrix} \vec{v}_{21}\\ \vec{v}_{22}\\ \vec{v}_{23} \end{bmatrix} = \begin{bmatrix} -1\\ 0\\ 1 \end{bmatrix}$ 
\end{center}.
$ $\\
$ $\\

Seguindo os mesmos passos mostrados anteriores, com $\lambda_3$ no lugar de $\lambda_2$, obtemos:

\begin{center}
    $\vec{v}_3 = \begin{bmatrix} \vec{v}_{31}\\ \vec{v}_{32}\\ \vec{v}_{33} \end{bmatrix} = \begin{bmatrix} 1\\ -2\\ 1 \end{bmatrix}$
\end{center}.

Então calculamos a decomposição espectral:

\begin{center}
    $S\Lambda S^{-1} = 
    \begin{bmatrix}
    0.5&0.25&0\\
    0.5&0.5&0.5\\
    0&0.25&0.5
    \end{bmatrix} = M $ \\
    $ $\\
    $ $\\
    
    $S \Lambda = \begin{bmatrix}
    1&-0.5&0\\
    2&0&0\\
    1&0.5&0
    \end{bmatrix} 
    \begin{bmatrix}
    0.25&0.25&0.25\\
    -0.5&0&0.5\\
    0.25&-0.25&0.25
    \end{bmatrix} = S^{-1}
    $\\
    $ $\\
    $ $\\
    $S\Lambda S^{-1} =
    \begin{bmatrix}
    1&-1&1\\
    2&0&-2\\
    1&1&1
    \end{bmatrix} 
    \begin{bmatrix}
    1&0&0\\
    0&0.5&0\\
    0&0&0
    \end{bmatrix}
    \begin{bmatrix}
    0.25&0.25&0.25\\
    -0.5&0&0.5\\
    0.25&-0.25&0.25
    \end{bmatrix}
    $
\end{center}.

Agora que temos a decomposição espectral, torna-se trivial o cálculo da proporção de probabilidade de genes, em qualquer período discreto.
$ $\\

Como por exemplo a distribuição de genes iniciais sendo $= \vec{v} = \begin{bmatrix} 1\\ 2\\ 3 \end{bmatrix}$. E o número de períodos $= 4$. \\

\begin{center}
    $M^4 \vec{v} = S\Lambda^4 S^{-1} \vec{v}$\\
    $ $\\
    $= \begin{bmatrix}
    1&-1&1\\
    2&0&-2\\
    1&1&1
    \end{bmatrix}
    \begin{bmatrix}
    1&0&0\\
    0&0.5&0\\
    0&0&0
    \end{bmatrix}^4
    \begin{bmatrix}
    0.25&0.25&0.25\\
    -0.5&0&0.5\\
    0.25&-0.25&0.25
    \end{bmatrix}
    \begin{bmatrix}
    1\\
    2\\
    3
    \end{bmatrix}
    $
    $ $\\
    $ $\\
    $ $\\
    
    $= \begin{bmatrix}
    1&-1&1\\
    2&0&-2\\
    1&1&1
    \end{bmatrix}
    \begin{bmatrix}
    1&0&0\\
    0&1/256&0\\
    0&0&0
    \end{bmatrix}
    \begin{bmatrix}
    0.25&0.25&0.25\\
    -0.5&0&0.5\\
    0.25&-0.25&0.25
    \end{bmatrix}
    \begin{bmatrix}
    1\\
    2\\
    3
    \end{bmatrix}
    $
    $ $\\
    $ $\\
    $ $\\
    $= \begin{bmatrix}
    383/256\\
    3\\
    385/256
    \end{bmatrix}$\\
    $ $\\
    $ $\\
    $= \begin{bmatrix}
    1.49609375\\
    3\\
    1.50390625
    \end{bmatrix}$
\end{center}


%%%%%%%%%%%%%%%%%%%%
\section{Resultados e Conclusão}
\label{sec:resultados}

Como resultado de nossos experimenos, podemos saber a chance das probabilidades de cada genótipo sobre a população, independente de qual for o período. Por meio do cálculo do autovetor sobre o autovalor dominante e da decomposição espectral da matriz $M$.\\

Os graficos abaixo representam respectivamente a mudança da probabilidade de cada tipo de gene em 4 períodos. com o $\vec{v}$ um vetor retirado aléatoriamente e em seguido normalizado. Vemos que a partir de um tempo $k$ sufuciente, os valorem tendem para um valor. Sendo os números desse valor, os número do autovalor dominante.
$ $\\

\begin{tikzpicture} %  indicação que estamos iniciando uma figura
\begin{axis}[ %  iniciando os eixos   
xlabel=período,            % nomeando os eixos
ylabel=probabilidade]
 
\addplot[color=blue,mark=*] coordinates{  % caracterizando a plotagem
(1, 1/15^0.5)
(2, 15^0.5 / 15)
(3, 15^0.5 / 12)
(4, 11*15^0.5/120)
(5, 23*15^0.5/240)
};
\end{axis}
\end{tikzpicture}

\begin{tikzpicture} %  indicação que estamos iniciando uma figura
\begin{axis}[ %  iniciando os eixos   
xlabel=período,            % nomeando os eixos
ylabel=probabilidade]
 
\addplot[color=blue,mark=*] coordinates{  % caracterizando a plotagem
(1,2/15^0.5)
(2,15^0.5 / 5)
(3, 15^0.5 / 5)
(4, 15^0.5 / 5)
(5, 15^0.5 / 5)
};
\end{axis}
\end{tikzpicture}

\begin{tikzpicture} %  indicação que estamos iniciando uma figura
\begin{axis}[ %  iniciando os eixos   
xlabel=período,            % nomeando os eixos
ylabel=probabilidade]
 
\addplot[color=blue,mark=*] coordinates{  % caracterizando a plotagem
(1, 3/15^0.5)
(2, 2 * 15^0.5 / 15)
(3, 7*15^0.5/60)
(4, 13*15^0.5 / 120)
(5,5 * 15^0.5 / 48)
};
\end{axis}
\end{tikzpicture}

%%%%%%%%%%%%%%%%%%%%%%%%%%%%%%
%%% referencias bibliograficas
\bibliographystyle{apalike}
% \bibliographystyle{abnt-alf}
\bibliography{biblio}
Araújo, Thelmo Pontes de, Álgebra linear: teoria e aplicações / Thelmo Pontes de Araújo. — Rio deJaneiro: SBM, 2014
\end{document}
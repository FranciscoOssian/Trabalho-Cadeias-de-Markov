\documentclass[a4paper,12pt]{article}
\usepackage[utf8]{inputenc}

% para separar silabas em portugues use
\usepackage[brazil]{babel}

% para indentar o primeiro parágrafo
\usepackage{indentfirst}

% para usar fontes e símbolos da AMS
\usepackage{amsfonts}
\usepackage{amssymb}
\usepackage{amsmath}

% para ambientes do tipo theorem
\newtheorem{proposition}{\scshape Proposição}[section]
\newtheorem{corollary}{\scshape Corolário}[section]
\newtheorem{lemma}{\scshape Lema}[section]
\newtheorem{definition}{\scshape Definição}[section]
\newtheorem{conjecture}{\scshape Conjectura}[section]
\newtheorem{example}{\scshape Exemplo}

% alguns atalhos para escrever vetores em negritoyhyhyhyhyhhyyhyhyhhy
\renewcommand{\u}{{\bf u}}
\renewcommand{\v}{{\bf v}}
\renewcommand{\sin}{\operatorname{sen}}
\providecommand{\w}{{\bf w}}
\providecommand{\x}{{\bf x}}
\providecommand{\y}{{\bf y}}
\providecommand{\q}{{\bf q}}
\providecommand{\bfa}{{\bf a}}
\providecommand{\bfb}{{\bf b}}
\providecommand{\bfc}{{\bf c}}
\providecommand{\zero}{{\bf 0}}
\providecommand{\spn}{\mathrm{span}}
\providecommand{\posto}{\mathrm{posto}}
\providecommand{\nul}{\mathrm{nul}}
\providecommand{\proj}{\mathrm{proj}}
\providecommand{\tr}{\mathrm{tr}}







%opening




\title{$2^o$ Trabalho de Álgebra Linear para Computação}
\author{
  Francisco Ossian Lopes Neto\\
  \and
  xxxxxxxxxxxxxxxxxxxxxxx\\
  \and
  yyyyyyyyyyyyyyyyyyyyyyyy\\
  \and
  zzzzzzzzzzzzzzzzzzzzzzzz\\
  \and
}
\date{abril 2020}

\begin{document}

\maketitle

\begin{abstract}
O resumo deve conter a descrição do problema, a técnica, algoritmo ou método utilizado, a metodologia e uma indicação dos resultados obtidos. Só pode haver um parágrafo no resumo e não é permitido fazer referências bibliográficasxxxxxxxxxxxxxxxxxxxxxxxxxxxxxxxxxxxx.
\end{abstract}


%%%%%%%%%%%%%%%%%%%%
\section{Introdução}
\label{sec:introducao}

Podemos abordar na probabilidade da genética, inumeros casos peculiares que podemos resolve-los atravez da Álgebra Linear. Como quais as proporções em que populações tendem depois de vários períodos, se conparadas pelos seus genes. Sabemos que existem genes que são mais dominates, e por isso exercem uma força maior na velocidade de reprodução. Tendo isto em questão, podemos usar o sistema das Cadeias de Markov junto a operações matriciais da Álgebra Linear para calcular a probabilidade da proporção de determinados grupos em tempos discretos. 


%%%%%%%%%%%%%%%%%%%%%%%%%%%%%%%
\section{Fundamentação Teórica (substitua por um título apropriado relacionado com seu trabalho)xxxxx}
\label{sec:fundamentacao}

Um processo de Markov é um processo estocástico em que a probabilidade de o sistema estar no estado $i$ no período $(n+1)$ depende somente do estado em que o sistema está no período $n$. Ou seja, para os processos de Markov, só interessa o estado imediato. Os principais elementos de um processo de Markov são dois:

\begin{itemize}

\item as probabilidades de transição $Mij$. Estas probabilidades de transição são normalmente agrupadas numa matriz, que denominamos matriz de transição, matriz estocástica ou ainda matriz de Markov. Onde $i$ representa o período $n$ e $j$ o período $n+1$. Ou $j$ o período $n$ e $i$ o período $n+1$

\item a probabilidade $x^i(n)$ de ocorrer o estado $i$ no $n$-ésimo período de tempo, ou, alternativamente, a fração da população em questão que está no estado $i$ no $n$-ésimo período de tempo.

\end{itemize}

    Para nós encontrarmos autovalores e autovetores de uma matriz $M$, precisamos seguir uma lógica de uma equação:
$M * \vec{v} = \lambda * \vec{v}$,
onde os simbolos representam respectivamente, a matriz de estados de Markov, um autovetor da matriz, um autovalor da matriz, e o autovetor da matriz. \\

    E, sendo um autorvetor $\vec{v}_2$ e um autovalor $\lambda_2$,
\begin{center}
$\lambda_2 I \vec{v}_2 = \lambda_2 \vec{v}_2$
\end{center}

    Onde $I$ é a matriz identidade.
Então podemos isolar a matriz de Markov e $\lambda I$, tal que.

\begin{center}
$(M-\lambda I)\vec{v} = \vec{0}$.\end{center}

Fazendo agora o determinate temos, 
$det((M-\lambda I)\vec{v}) = det(\vec{0})$, logo:

\begin{center}
    $det(M - \lambda I) = 0$
\end{center}

Pois por definição $\vec{v}$ é não nulo, e $det(0) = 0$\\\\
E para conseguir os autovalores, fazemos o determinate, tendo a matriz $M$. No final encontramos um polinômio de grau 3,  $ax^3 + bx^2 + cx^1 + dx^0$. Onde cada raíz deste polinômio é um $\lambda$ (autovalor) da matriz M.\\

E em casos que $\lambda$ é um valor entre $0$ e $1$, com o passar de $n$ períodos, onde $n$ tende a infinito, o resesultado irá tender a $0$, pois\\
\begin{center}
$\lim_{n\to\infty} x^n = 0$\\ tal que,  $0 < n < 1$\end{center}

E calculando o autovalor, utilizamos mais uma vez\\
\begin{center}
    $M \vec{v} = \lambda \vec{v}$,
\end{center}
e substituimos $\lambda$ pelo valor que encontramos, formando um sistema e encontrando o autovetor $\vec{v}$.

%%%%%%%%%%%%%%%%%%%%%
\section{Metodologia e Experimentos}
\label{sec:metodologia}


Consideremos uma matriz $M$, tal que $Mij$ representa a transição do estado $i$ para o estado $j$. Fechando exclusivamente a uma transição em que temos no promeiro período um genótipo qualquer e nos $N$ proxímos períodos esse genótipo qualquer seja combinado com um genótipo heterogêneo. Tendo assim apenas 1 probabilidade para cada transição. 

\begin{displaymath}
M = \begin{bmatrix}
0.5&0.25&0\\
0.5&0.5&0.5\\
0&0.25&0.5
\end{bmatrix}
\end{displaymath}

Os experimentos consistem em Encontrar respostas de $n$ períodos das transições de estados, por meio de técnicas matriciais, como autovalores, autovetores e decomposição espectral.

Os autovalores iram simplificar os cálculos da potência da matriz $M$, por meio da diagonalização da mesma. 

O autovalor predominante da matriz $M$ vai nos dar por conseguinte seu autovetor, sendo este autovetor a proporção de cada genótipos no periodo $n$.

Calculando os autovalores, temos:
\begin{center}
    $\lambda_1 = 0$,\\ $\lambda_2$ = 0.5,\\ $\lambda_3 = 1$.
\end{center}

Pois,
\begin{center}
    $det(M-\lambda I) = 0$\\
    $= (0.5 - \lambda)^3 - 1/16 - \lambda/8 = 0$
\end{center}
e $\lambda_1$, $\lambda_2$, $\Lambda_3$ são suas respectivas raízes

Como já definido, se um autovalor for entre $0$ e $1$, ele não será considerado um autovalor suficiente para $n$ tendendo ao infinito.

Que no caso da matriz $M$, os limites dos autovalores são :
\begin{center}
    $\lim_{k\to\infty} (\lambda_1)^k = 0$,\\
    $\lim_{k\to\infty} (\lambda_2)^k = 0$,\\
    $\lim_{k\to\infty} (\lambda_3)^k = 1$
\end{center}

Então o autovalor dominante é o $\lambda_3$. E seu autovetor ascossiado é, no caso ao subistituirmos $\lambda_3$ em $M\vec{v} = \lambda \vec{v}$ e calcularmos o sistema objtido, obtemos que o autovetor $\vec{v}$ ascosicado a $\lambda_3$ é igual a:

\begin{displaymath}
\vec{v_3} = \begin{bmatrix}
1\\
2\\
1
\end{bmatrix}
\end{displaymath}

Sendo está a proporção de genes quando o número de períodos tende a infinito.
Mas esta proporção não está representada como porcentagem. Para nós convertermos a proporção de genes para porcentagem, precisamos normalizar o vetor $\vec{v}_3$ para norma $1$, assim a soma de suas porcentagens/linhas será igual a $1$, logo:
\begin{center}
    $||\vec{v}_3|| = \dfrac{\vec{v}_3}{\langle\ \vec{v}_3,\vec{v}_3 \rangle}$
    $ = \begin{bmatrix} \dfrac{1}{\sqrt{6}}\\ \dfrac{2}{\sqrt{6}}\\ \dfrac{1}{\sqrt{6}} \end{bmatrix}$
\end{center}




xxxxxxxxxxxxxNesta seção, todo a metodologia usada no trabalho deve ser explicitada. Ou seja, a sequência do que foi feito e os parâmetros usados.

Por exemplo, a matriz de Leslie relativa à população feminina americana em 1967.

Os experimentos consistem, por exemplo, da centralização dos dados na matriz $X$, as matrizes formadas para compor as imagens (equipe C).xxxxxxxxxxxx



%%%%%%%%%%%%%%%%%%%%
\section{Resultados e Conclusão}
\label{sec:resultados}

Nesta seção, aparecem os gráficos, as tabelas etc. Lembre-se que todo gráfico (que estão num ambiente de figura) ou tabela deve ser referenciado e explicado no \textbf{texto}.

A conclusão é um pequeno resumo dos resultados obtidos, por exemplo, como a população de genes dominantes, híbridos e recessivos se estabiliza.

%%%%%%%%%%%%%%%%%%%%%%%%%%%%%%
%%% referencias bibliograficas
\bibliographystyle{apalike}
% \bibliographystyle{abnt-alf}
\bibliography{biblio}
Araújo, Thelmo Pontes de, Álgebra linear: teoria e aplicações / Thelmo Pontes de Araújo. — Rio deJaneiro: SBM, 2014
\end{document}